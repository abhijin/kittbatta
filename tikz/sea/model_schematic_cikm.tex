\PassOptionsToPackage{usenames,dvipsnames}{xcolor}
\documentclass[tikz,border=2]{standalone}
\usepackage[sfdefault]{noto}
\usetikzlibrary{decorations.pathreplacing,shadows,arrows,shapes,positioning,calc,backgrounds,fit,automata,shadows}
\usepackage{xparse}
\usepackage{ifthen}
%\usepackage{xifthen}
\ExplSyntaxOn
\NewDocumentCommand { \xifnum } { }
    {
        \fp_compare:nTF
    }
\ExplSyntaxOff

\definecolor{RED}{HTML}{E41A1C}
\definecolor{BLUE}{HTML}{377EB8}
\definecolor{GREEN}{HTML}{4DAF4A}
\definecolor{ORANGE}{HTML}{FF7F00}



\usepackage{comment}

\tikzset{cross/.style={cross out, draw, 
         minimum size=2*(#1-\pgflinewidth), 
         inner sep=0pt, outer sep=0pt}}
\begin{document}

\newcommand{\asd}{\alpha_s}
\newcommand{\afm}{\alpha_{\ell}}
\newcommand{\ald}{\alpha_{\ell d}}

\newcommand{\layersep}{1cm}
\newcommand*{\arrowcolor}{black}%
\newcommand*{\infcolor}{Red}%
\newcommand*{\suscolor}{Green}%
\newcommand*{\loccolor}{black!80}%
\newcommand*{\moorecolor}{\loccolor}%
\newcommand*{\loccolora}{Green!40!black}%
\newcommand*{\loccolorb}{Green!40!black}%
\newcommand*{\locline}{black!50}%
\newcommand*{\backcolor}{black}%
\newcommand*{\backopacity}{0.2}%
\newcommand*{\backline}{white}%
\newcommand*{\arrowopacity}{1}%
\newcommand*{\radcolor}{white}%

\def\MooreNbd#1#2#3#4#5{
    \newcommand*{\xycolor}{#1}%
    \newcommand*{\xMin}{#2}%
    \newcommand*{\xMax}{#3}%
    \newcommand*{\yMin}{#4}%
    \newcommand*{\yMax}{#5}%
    \pgfmathparse{\xMin + 0.5}
    \pgfmathresult \let\xpoint\pgfmathresult;
    \pgfmathparse{\yMin + 0.5}
    \pgfmathresult \let\ypoint\pgfmathresult
    \foreach \x in {\xMin,\xpoint,...,\xMax}
    	{\foreach \y in {\yMin,\ypoint,...,\yMax}
            {\draw[draw=\locline,fill=\xycolor] (\x,\y) rectangle (\x + 0.5,\y + 0.5);}}
    
}

\def\Locality#1#2#3#4#5{
        \newcommand*{\abcolor}{#1}%
        \newcommand*{\aMin}{#2}%
        \newcommand*{\aMax}{#3}%
        \newcommand*{\bMin}{#4}%
        \newcommand*{\bMax}{#5}%
        \pgfmathparse{\aMin + 0.5}
        \pgfmathresult \let\apoint\pgfmathresult;
        \pgfmathparse{\bMin + 0.5}
        \pgfmathresult \let\bpoint\pgfmathresult
        \foreach \a in {\aMin,\apoint,...,\aMax} 
        	{\foreach \b in {\bMin,\bpoint,...,\bMax}
                {\xifnum{\a!=\aMin}
                	{\xifnum{\b!=\bMin}
                	    {\xifnum{\b!=\bMax}
                        {\draw[draw=\locline,fill=\abcolor] (\a,\b) rectangle (\a + 0.5,\b + 0.5);}
                            {\xifnum{\a!=\aMax}
                            {\draw[draw=\locline,fill=\abcolor] (\a,\b) rectangle (\a + 0.5,\b + 0.5);}           
                                {}}}
                        {\xifnum{\a!=\aMax}
                            {\draw[draw=\locline,fill=\abcolor] (\a,\b) rectangle (\a + 0.5,\b + 0.5);}
                            {}}}
                    {\xifnum{\b!=\bMin}
                	    {\xifnum{\b!=\bMax}
                            {\draw[draw=\locline,fill=\abcolor] (\a,\b) rectangle (\a + 0.5,\b + 0.5);}           {}}
                        {}}}
                        }
}

\def\LocalityTwo#1#2#3#4#5{
        \newcommand*{\cdcolor}{#1}
        \newcommand*{\cMin}{#2}%
        \newcommand*{\cMax}{#3}%
        \newcommand*{\dMin}{#4}%
        \newcommand*{\dMax}{#5}%
        \pgfmathparse{\cMin + 0.5}
        \pgfmathresult \let\cpoint\pgfmathresult;
        \pgfmathparse{\dMin + 0.5}
        \pgfmathresult \let\dpoint\pgfmathresult
        \foreach \c in {\cMin,\cpoint,...,\cMax} 
        	{\foreach \d in {\dMin,\dpoint,...,\dMax}
                {\xifnum{\c!=\cMin}
                	{\xifnum{\d!=\dMin}
                	    {\xifnum{\d!=\dMax}
                        {\draw[draw=\locline,fill=\cdcolor] (\c,\d) rectangle (\c + 0.5,\d + 0.5);}
                            {\xifnum{\c!=\cMax}
                            {\draw[draw=\locline,fill=\cdcolor] (\c,\d) rectangle (\c + 0.5,\d + 0.5);}           
                                {}}}
                        {\xifnum{\c!=\cMax}
                            {\draw[draw=\locline,fill=\cdcolor] (\c,\d) rectangle (\c + 0.5,\d + 0.5);}
                            {}}}
                    {\xifnum{\d!=\dMin}
                	    {\xifnum{\d!=\dMax}
                            {\draw[draw=\locline,fill=\cdcolor] (\c,\d) rectangle (\c + 0.5,\d + 0.5);}           {}}
                        {}}}
                        }
                        }





\begin{tikzpicture}[scale=.7,every node/.style={minimum size=1cm},on grid, block/.style ={rectangle, draw=black, thick, text width=3mm,
align=center, rounded corners, minimum height=10mm, minimum width = 7mm},
dedge/.style={ultra thick, black!60, >=latex', shorten >=.5pt, shorten <=.5pt}]

		
    %slanting: production of a set of n 'laminae' to be piled up. N=number of grids.
    \begin{scope}[
            yshift=-83,
            every node/.append style={yslant=0.5,xslant=-1},
            yslant=0.5,xslant=-1
            ]
        % opacity to prevent graphical interference
        \fill[draw=\backcolor,fill opacity=\backopacity] (0,0) rectangle (7,7);
        \draw[step=5mm, \backline] (0,0) grid (7,7); %defining grids
        
        
        % {<color of outside cells>} {<xmin>} {<xmax>} {<ymin>} {<ymax>}
        \MooreNbd{\moorecolor} {0} {1} {0} {1}
        %Infected cell
        \fill[\suscolor] (0.5,0.5) rectangle (1,1);
        \fill[\infcolor](0,1) rectangle (0.5, 1.5);
        \fill[\infcolor](1,0) rectangle (1.5, 0.5);
         
         
         
        \Locality{\loccolora} {4.5} {6.5} {3.5} {5.5}
        %City
        \fill[\suscolor] (5,4) rectangle (5.5, 4.5);
        %Infected cell
        \fill[\infcolor] (4.5,4.5) rectangle (5,5);
        \fill[\infcolor] (5.5,3.5) rectangle(6,4);
        \fill[\infcolor] (5.5,5) rectangle (6,5.5);
        
     
        
        \LocalityTwo{\loccolorb} {0} {2} {2.5} {4.5}
        %City
        \fill[\suscolor] (1,3.5) rectangle (1.5, 4);
        
        %Diagram
		\fill[\suscolor](6.5,0) rectangle (7,0.5); 
		\fill[\infcolor](5.5,1) rectangle(6,1.5);       
        
	    \draw[\backline,very thick] (0,0) rectangle (7,7);%marking borders
        
        %Radius
        \draw[\radcolor,thick,dashed] (1.25,3.75) circle (30pt);
    \end{scope}

    %% %putting arrows and labels:
    %% \draw[-latex,thick,blue] (7.5,2) node[right]{Farm to Market Spread}
    %%      to (7,2);

    %% \draw[-latex,thick,red](7.5,1)node[right]{Short Distance Spread}
    %%     to (7,1);

    %% \draw[-latex,thick,purple](7.5,3)node[right]{Market to Market Spread}
    %%     to (7,3);

	%Cell properties diagram
	
%	\draw[-latex,thick,black](5.5,1.3)to(5.5,3)to(7,3)to node[above]{$\rho$}(7,3.5);
%	\draw[-latex,thick,black](5.5,3)to node[above]{$\epsilon$}(5.5,3.5);
%	\draw[-latex,thick,black](5.5,3)to(4,3)to node[above]{SEI}(4,3.5);


%susceptible
\draw[-latex,thick,black](6.5,0.8)to(6.5,1.5)to node[above]{$\epsilon (v,t)$}(6.5,2.5);


%arrow
    \draw[-latex,thick,\arrowcolor, opacity = \arrowopacity](4.5,0.4)node[right]{}
        to[out=-70,in=-120] (6.5,0.4);
        
        %infected
        \draw[-latex,thick,black](4.5,0.8)to(4.5,1.3)to
        node[above=1mm]{$\rho (v,t)$}(4.5,2.5);
        
%%         %sei
%%         \node[block,fill=\suscolor] at (2, -3)(S){S};
%%         \node[block,fill=yellow] at (4.3, -3)(E){E} ;
%%         \node[block,fill=\infcolor] at (6.5,-3) (I){I};
%%         \draw[->,dedge] ($(S.north east)+(0,-1.3)$) -- ($(E.north west)+(0,-1.3)$) 
%% node[midway,above=-3.8mm,black,text width=2.5cm,align=center](ald){$\ald$};
%% 	\draw[->,dedge] (S) -- (E) 
%% node[midway,above=-3.8mm,black,text width=2.5cm,align=center](afm){$\afm$};
%% \draw[->,dedge] ($(S.south east)+(0,1.3)$) -- ($(E.south west)+(0,1.3)$) 
%% node[midway,above=-3.8mm,black,text width=2.5cm,align=center](asd){$\asd$};
%%  
%% %,$\afm$,$\ald$};
%% %\node[above=-2mm] at (weights) {Weights};
%% 
%%         \draw[->,dedge] (E) -- (I) node[midway,above=-3.5mm,black] (latency){$\ell$};
        
        %\node[above=-2mm] at (latency) {Latency};
       %suscolor!60

%\node[block,fill=GREEN!50] (S) {S};
	
	%Key
	%Spread
	%% \draw[-latex,thick,\arrowcolor](-9,6)to node[right]{}(-8.5,6);
	%% \node[right] at (-8.46, 6){Spread};
	
        \begin{scope}[shift={(0,.5)}]
	%Sum
	\draw[black](-8.75,5.5) circle(5pt);     
	\draw (-8.75,5.5) node[cross=2pt,rotate=45,black]{};  
	\node[right] at (-8.53,5.5){Sum}; 
	
	%Infected cell
	\draw[fill=\infcolor, draw=\infcolor] (-8.9,4.8) rectangle 		
		(-8.6,5.1);
		\node[right] at (-8.5, 4.9){Infected node};
	
	%Susceptible cell
	\draw[fill=\suscolor, draw=\suscolor] (-8.9,4.2) rectangle 
	 	(-8.6,4.5);
		\node[right] at (-8.5, 4.3){Target susceptible node};

		%Switch comments for different styles
%		%Neighborhood
%		\draw[fill=\loccolor, draw=\loccolor] (-8.9,3.6) rectangle 
%		(-8.6,3.9);
%		\node[right] at (-8.5, 3.72){Neighborhood};


		%Neighborhood
	 	\draw[fill=\loccolor, draw=\loccolor] (-8.9,3.6) rectangle 
		(-8.6,3.9);
		\node[right] at (-8.5, 3.72){Moore neighborhood};
    \end{scope}
	
	%Arrows
	%Farm to Market
    \draw[-latex,thick,\arrowcolor, opacity = \arrowopacity](0,1.9)node[right]{}
        to[out=70,in=160] (1,1.8);
    \draw[-latex,thick,\arrowcolor, opacity = \arrowopacity](1.9,1.7)node[right]{}
        to[out=110,in=20] (1,1.8);
    \draw[-latex,thick,\arrowcolor, opacity = \arrowopacity](0.4,2.5)node[right]{}
        to[out=20,in=90] (1,1.8);
        
        %Short distance
    \draw[-latex,thick,\arrowcolor, opacity = \arrowopacity](-0.8,-2.2)node[right]{}
        to[out=90,in=110] (0,-2.2);
    \draw[-latex,thick,\arrowcolor, opacity = \arrowopacity](0.8,-2.2)node[right]{}
        to[out=90,in=70] (0,-2.2);
        
        %Market to Market
    
	\draw[black](0.6,5) circle(10pt);     
	\draw (0.6,5) node[cross=4pt,rotate=45,black]{};   
        
    \draw[-latex,ultra thick,\arrowcolor, opacity = \arrowopacity](0.2,5.2)node[right]{}
        to[out=140,in=70] (-2.7,3.9);
        
	\draw[-latex,thick, \arrowcolor, opacity = \arrowopacity](0,1.9)node[right]{}to(0.5,4.6);
	\draw[-latex,thick, \arrowcolor, opacity = \arrowopacity](1.9,1.7)node[right]{}to(0.7,4.6);
	\draw[-latex,thick, \arrowcolor, opacity = \arrowopacity](0.4,2.5)node[right]{}to(0.6,4.6);

    \draw[black](-2.7,3.5) circle(10pt);     
	\draw (-2.7,3.5) node[cross=4pt,rotate=45,black]{};  
	
	\draw[-latex,thick,\arrowcolor, opacity = \arrowopacity](-2.7,3.1)node[right]{}to(-3.5,-0.4);
	\draw[-latex,thick,\arrowcolor, opacity = \arrowopacity](-2.7,3.1)node[right]{}to(-2.5,-0.5);
	\draw[-latex,thick,\arrowcolor, opacity = \arrowopacity](-2.7,3.1)node[right]{}to(-1.9,0.3);


%% Label arrows
    %%\draw[-latex,thick,black](-3.5,-2)node[left]{Moore neighborhood}
    %%     to (-1.5,-2.25);
         
        \draw[-latex,thick,black](-5.3,-1)node[left]{Group}
         to (-3.9,-0.6);
         
         \draw[-latex,thick](1,6)node[text
         width=3.7cm,right,black](loc){Spread across groups}
         to (-0.5,5.6);
         
         \draw[-latex,thick](1.5,4.2)node[text
         width=3.7cm,anchor=west,black](loc){Spread within a group} to (1,2.2);
         
         \draw[-latex,thick](-1.5,-2.7)node[text width=4cm,anchor=east
         ,black](loc){Short-distance spread}
         to (-0.5,-2);
	
\node at (-1,6){$F_{ij}(t)$};
	
\end{tikzpicture}
\end{document}
