\PassOptionsToPackage{table,usenames,dvipsnames}{xcolor}
\documentclass[tikz,border=2]{standalone}
\usetikzlibrary{shadows,arrows,arrows.meta,bending,shapes,positioning,calc,backgrounds,fit,automata}
\usetikzlibrary{decorations.text}
\usepackage{varwidth}
\usepackage[scaled]{libertine}
%%\usepackage[scaled]{helvet}
\renewcommand{\familydefault}{\sfdefault} 
%% \usepackage{lmodern} % enhanced version of computer modern
\usepackage[T1]{fontenc} % for hyphenated characters and textsc in section title
\usepackage{amssymb}
\usepackage{mathtools} % contains amsmath which comes with align
\usepackage{amsthm}
\usepackage{graphicx}
\graphicspath{{aux/}}
\usepackage{microtype} % some compression
\usepackage[skins]{tcolorbox}
\begin{document}
%%%%%%%%%%%%%%%%%%%%%%%%%%%%%%%%%%%%%%%
% Define the layers to draw the diagram
\pgfdeclarelayer{bg}
\pgfsetlayers{bg,main}
%%%%%%%%%%%%%%%%%%%%%%%%%%%%%%%%%%%%%%%
% colors
\definecolor{}{HTML}{D8392B}
\definecolor{DarkRed}{HTML}{D8392B}
\definecolor{DarkBlue}{HTML}{558B92}
\definecolor{DarkBrown}{HTML}{62524F}
\definecolor{DarkGreen}{HTML}{A18F6A}
\definecolor{Red}{HTML}{E65942}
\definecolor{Blue}{HTML}{78A4AC}
\definecolor{Brown}{HTML}{7B6663}
\definecolor{Green}{HTML}{C4AE87}
\definecolor{Grey}{HTML}{D3D3D3}
%%%%%%%%%%%%%%%%%%%%%%%%%%%%%%%%%%%%%%%
% wheel chart stuff
%%%%%%%%%%%%%%%%%%%%%%%%%%%%%%%%%%%%%%%
% Adjusts the size of the wheel:
\newcommand{\ring}[5]{
\def\outerradius{#2}
\def\innerradius{#3}
\pgfmathsetmacro{\cumnum}{#4}
\pgfmathsetmacro{\totalangle}{#5}
\def\segmentgapfraction{5}

\pgfmathsetmacro{\totalnum}{0}
% Calculate the thickness and the middle line of the wheel
\pgfmathsetmacro{\midradius}{(\outerradius+\innerradius)/2}

\foreach \value/\colour/\name/\label in {#1} {
    \pgfmathparse{\value+\totalnum}
    \global\let\totalnum=\pgfmathresult}

% Loop through each value set. \cumnum keeps track of where we are in the wheel
\foreach \value/\colour/\name/\label in {#1} {
    \pgfmathsetmacro{\newcumnum}{\cumnum + \value/\totalnum*\totalangle}
    \pgfmathsetmacro{\truncnewcumnum}{\newcumnum-\segmentgapfraction}
    
    % Calculate the mid angle of the colour segments to place the labels
    \pgfmathsetmacro{\midangle}{-(\cumnum+\newcumnum)/2}
    
    % This is necessary for the labels to align nicely
    \pgfmathparse{(\cumnum>90?1:0)} \edef\westind{\pgfmathresult}
    \pgfmathparse{(\cumnum<270?(\westind*1):0)} \edef\westind{\pgfmathresult}
    \pgfmathparse{((\westind==1)?"east":"west")} \edef\textanchor{\pgfmathresult}
    \pgfmathsetmacro{\labelshiftdir}{1-2*(\westind)}
    
    % Draw the color segments. Somehow, the \midrow units got lost, so we add 'pt' at the end. Not nice...
    \fill[\colour,opacity=1] (\cumnum:\outerradius) arc (\cumnum:\truncnewcumnum:\outerradius) 
        -- (\truncnewcumnum:\innerradius) arc (\truncnewcumnum:\cumnum:\innerradius) 
        -- cycle;
    
    % Draw the data labels
        \draw  [\colour,thin,text=black] node [append after command={(\cumnum:\innerradius)
    -- (\cumnum:\outerradius + 2ex) 
    -- (\tikzlastnode.\textanchor)}] at
    (\cumnum:\outerradius + 2ex) [xshift=\labelshiftdir*.2cm,
anchor=\textanchor,text=DarkBrown]{\small \name};
    
    % Set the old cumulated angle to the new value
    \global\let\cumnum=\newcumnum
}
}

\newcommand{\circtext}[7]{
\def\outerradius{#2}
\def\innerradius{#3}
\def\midradius{#4}
\pgfmathsetmacro{\cumnum}{#5}
\pgfmathsetmacro{\totalangle}{#6}
\pgfmathsetmacro{\opacity}{#7}
\def\segmentgapfraction{5}

\pgfmathsetmacro{\totalnum}{0}
% Calculate the thickness and the middle line of the wheel
%%\pgfmathsetmacro{\midradius}{\outerradius/2+\innerradius/2}

\foreach \value/\colour/\name/\textcolor in {#1} {
    \pgfmathparse{\value+\totalnum}
    \global\let\totalnum=\pgfmathresult}

% Loop through each value set. \cumnum keeps track of where we are in the wheel
\foreach \value/\colour/\name/\textcolor in {#1} {
    \pgfmathsetmacro{\newcumnum}{\cumnum + \value/\totalnum*\totalangle}
    \pgfmathsetmacro{\truncnewcumnum}{\newcumnum-\segmentgapfraction}
    
    % Calculate the mid angle of the colour segments to place the labels
    \pgfmathsetmacro{\midangle}{-(\cumnum+\newcumnum)/2}
    
    % This is necessary for the labels to align nicely
    \pgfmathparse{(\cumnum>90?\cumnum:\truncnewcumnum)} \edef\startangle{\pgfmathresult}
    \pgfmathparse{(\cumnum>90?\truncnewcumnum:\cumnum)} \edef\endangle{\pgfmathresult}
    
    % Draw the color segments. Somehow, the \midrow units got lost, so we add 'pt' at the end. Not nice...
    \fill[\colour,opacity=\opacity,thick] (\cumnum:\outerradius) arc (\cumnum:\truncnewcumnum:\outerradius) 
        -- (\truncnewcumnum:\innerradius) arc (\truncnewcumnum:\cumnum:\innerradius) 
        -- cycle;
    
    % Draw the data labels
        \draw [draw=none, font=\small,thick,postaction={decorate,decoration={text along path,text
        align=center,text color=\textcolor,text={\name}}}] (\startangle:(\midradius) arc
    (\startangle:\endangle:\midradius);

    
    % Set the old cumulated angle to the new value
    \global\let\cumnum=\newcumnum
}
}

\newcommand{\thickarrow}[5]{
    \begin{scope}[transparency group, opacity=0.75]
        \draw[{[bend] Round Cap[length=1pt]}-{[bend] Triangle
        Cap[length=7pt,width=12.5pt]},line width=4pt,#5!50] ($(#1)$) to [#4]
        ($(#2)+(#3)$);
    \end{scope}
}

\begin{tikzpicture}[
        dedge/.style={line width=1mm, black!60, arrows={
    -Latex[]},opacity=.5}]
    \ring{
        1/Brown/\parbox{2.6cm}{1.1~Consignment \\inspection design}/, 
        1/Brown/\parbox{2cm}{1.2~Surveillance in the field}/, 
        1/Brown/\parbox{2cm}{1.3~Robust\\ surveillance for multiple entry
        scenarios}/,
        1/Blue/\parbox{2cm}{2.4~Multi-mode risk-aware surveillance}/, 
        1/Blue/\parbox{1.6cm}{2.3~Adaptive monitoring}/, 
        1/Blue/\parbox{2.5cm}{2.2~Inferring model from range maps}/,
        1/Blue/\parbox{1.6cm}{2.1~Inferring the state of the spread}/,
        1/Green/Diverse datasets/,
        1/Green/Spread models/
    }{4cm}{3.9cm}{120}{360}
    %%
    \circtext
    {3/Brown/Topic 1: Surveillance for early discovery/Brown, 
    4/Blue/Topic 2: Inferring from and responding to spread/Brown}
    {3.8cm}{3.35cm}{3.7cm}{120}{280}{.2}
    \circtext{
    1/Red/Optimization/white}{3.3cm}{2.9cm}{3.2cm}{120}{200}{1}
    \circtext{
    1/Brown/Inference/white}{2.9cm}{2.4cm}{2.8cm}{280}{120}{1}
    \circtext
    {1/Green/Data and Evaluation/Brown}{3.8cm}{2.9cm}{3.3cm}{40}{80}{.2}

%% dependencies
%%%% first nodes are drawn:
    \def\rad{2.8cm}
\node (11) [] at (140:\rad) {};
\node (12) [] at (180:\rad) {};
\node (13) [] at (220:\rad) {};
\node (24) [] at (260:\rad) {};
\node (23) [] at (300:2.3cm) {};
\node (22) [] at (340:2.3cm) {};
\node (21) [] at (20:2.3cm) {};
%%%% edges
\thickarrow{11}{21}{-.2,.4}{bend right}{Brown}
\thickarrow{11}{24}{.1,0}{bend left}{Brown}
\thickarrow{12}{13}{-.2,.2}{bend left=70}{Brown}
\thickarrow{12}{21}{-.1,.2}{bend right}{Brown}
\thickarrow{12}{22}{0,0}{bend right}{Brown}
\thickarrow{12}{24}{0,0}{bend left}{Brown}
\thickarrow{21}{23}{0,0}{bend right}{Blue}
\thickarrow{22}{11}{0,0}{bend left}{Blue}
\thickarrow{22}{12}{0,.2}{bend right}{Blue}
\thickarrow{22}{13}{0,0}{bend right}{Blue}
\thickarrow{22}{21}{.2,-.4}{bend left=70}{Blue}
\end{tikzpicture}
\end{document}
\end{tikzpicture}
\end{document}
