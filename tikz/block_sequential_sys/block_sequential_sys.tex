\documentclass[tikz,border=2]{standalone}
\usetikzlibrary{shadows,arrows,shapes,positioning,calc,backgrounds,fit}
\newcommand{\vanish}[1]{}
% Define the layers to draw the diagram
%
\begin{document}
\pgfdeclarelayer{bg}
\pgfdeclarelayer{fg}
\pgfsetlayers{bg,main,fg}
%
\begin{tikzpicture}
[scale=2,node distance=1cm, transform shape,
every node/.style={shape=circle,draw=black,fill,scale=.5},
myblock/.style={draw=none,fill=none,scale=1.5,inner sep=0pt},
myedge/.style={semithick},
every fit/.style={fill=black!10,ellipse,draw=black!20},
myellipse/.style={fill=black,draw=none,opacity=.1}]
%%%%%%%%%%
\begin{scope}[xshift=3cm,yshift=3cm,rotate=130]
\def \n {4}
\def \rad {.75cm}
\foreach \x in {1,...,4}
\draw node(a\x) at ({360/\n * \x}:\rad) {\vanish{a\x}};
\draw[myedge]
(a1)--(a2)--(a3)--(a4)--(a1)--(a3);
\end{scope}
%%%%%%%%%%
\begin{scope}[xshift=1cm,yshift=2cm]
\def \n {5}
\def \rad {1cm}
\foreach \x in {1,...,5}
\draw node(b\x) at ({360/\n * \x}:\rad) {\vanish{b\x}};
\draw[myedge]
(b1)--(b2)--(b3)--(b4)--(b5)--(b1)--(b3)--(b5)--(b2)--(b4)--(b1);
\end{scope}
%%%%%%%%%%
\begin{scope}[xshift=4.5cm,yshift=1.5cm]
\def \n {3}
\def \rad {.75cm}
\foreach \x in {1,...,3}
\draw node(c\x) at ({360/\n * \x}:\rad) {\vanish{b\x}};
\draw[myedge] (c1)--(c2)--(c3)--(c1);
\end{scope}
%%%%%%%%%%
\draw node (d1) at (3,.5) {};
%%%%%%%%%%
%% stitching together
\draw[myedge] (d1) -- (c2);
\draw[myedge] (d1) -- (b3);
\draw[myedge] (c1) -- (a3);
\draw[myedge] (b1) -- (a4);
\draw[myedge] (b5) -- (a1);
%%
\begin{pgfonlayer}{bg}
\draw[myellipse] (b1) circle[xshift=-.3cm,yshift=-.95cm,radius=1.15cm];
\draw[myellipse] (a1) circle[xshift=.6cm,yshift=.5cm,radius=.9cm];
\draw[myellipse] (c2) circle[xshift=.35cm,yshift=.65cm,radius=.9cm];
\draw[myellipse] (d1) circle[xshift=.0cm,yshift=.0cm,radius=.4cm];
\end{pgfonlayer}
%%
\node[myblock] at (1,2) {$B_1$};
\node[myblock] at (3,3.3) {$B_2$};
\node[myblock] at (4.5,1.5) {$B_3$};
\node[myblock] at (3.2,1.1) {$B_4$};
\end{tikzpicture}
\end{document}
